\documentclass[a4paper, 11pt]{article}


% --- Packages ---
\usepackage[utf8]{inputenc}
\usepackage[T1]{fontenc}
\usepackage{geometry}
\geometry{margin=1in}     % Standard margins
\usepackage{graphicx}     % For images
\usepackage{float}        % To force image placement
\usepackage{xcolor}       % For colors
\usepackage{hyperref}     % For clickable links
\usepackage{listings}     % For code and API requests
\usepackage{booktabs}     % For nicer tables
\usepackage{fdsymbol}     % For suit symbols
\usepackage{colortbl}     % For table highlighting



% --- Required Packages ---
\usepackage{xcolor}     % Required for the red color
\usepackage{booktabs}   % Required for the tables (\toprule, \midrule)

% --- FIXED Custom Suit Commands ---
% These now use \ensuremath so they won't break your document
\newcommand{\redheart}{\textcolor{red}{\ensuremath{\varheartsuit}}}
\newcommand{\reddiamond}{\textcolor{red}{\ensuremath{\vardiamondsuit}}}
\newcommand{\blackclub}{\ensuremath{\clubsuit}}
\newcommand{\blackspade}{\ensuremath{\spadesuit}}


% --- Code Block Styling ---
\definecolor{codegray}{rgb}{0.95,0.95,0.95}
\definecolor{codeblue}{rgb}{0.0,0.0,0.6}

\lstset{
    backgroundcolor=\color{codegray},
    basicstyle=\ttfamily\small,
    commentstyle=\color{gray},
    keywordstyle=\color{codeblue},
    breaklines=true,
    frame=single,
    captionpos=b,
    keepspaces=true,
    columns=flexible,
    showstringspaces=false
}

% --- Document Info ---
\title{\textbf{ASE Project Report - GROUP 2}}
\date{\today}

\begin{document}

\thispagestyle{empty} % Removes page number from the title page

% --- 1 Page: Names of members and group (if you want also a table of contents).
\begin{center}
    \vspace*{1cm}
    {\Huge \textbf{ASE Project Report - GROUP 2} \par}
    \vspace{0.5cm}
    {\Large \textbf{Advanced Software Engineering} \par}
    \vspace{2cm}

    \begin{table}[h]
        \centering
        \begin{tabular}{@{}cccc@{}}
            \toprule
            \textbf{First Name} & \textbf{Last Name} & \textbf{Student ID} & \textbf{E-Mail}\\
            \midrule
            Filippo & Morelli & 608924 & f.morelli38@studenti.unipi.it\\
            Federico & Fornaciari & 619643 & f.fornaciari@studenti.unipi.it\\
            Ashley & Spagnoli & 655843 & a.spagnoli9@studenti.unipi.it\\
            Marco & Pernisco & 683674 & m.pernisco@studenti.unipi.it\\
            \bottomrule
        \end{tabular}
    \end{table}
\end{center}

\vspace{1cm}
{
    \hypersetup{linkcolor=black} % Optional: Keeps TOC black if using hyperref
    \tableofcontents
}

\newpage

%We used Centralized Auth (Access Security Lab)
%For input sanitization it was enough to check that the input was a string, since MongoDB isn't vulnerable from string concatenation attacks but only from dictionaries in the input interpreted by jsonify
%History app_test fills up with 2 mock endpoints

% --- 1-2 Page: Cards overview (some images, short description, etc.).
\section{Cards Overview}
As cards we used the classical 52 + Joker deck.\\
As anyone could expect, the two features assigned to those cards are the number and the suit, while the Joker is a special card with no features, that beats any other card.
\begin{figure}[H]
    \centering
    \includegraphics[width=0.8\textwidth]{cards.png}
    \label{fig:arch}
\end{figure}


% 1-3 Pages: Architecture (details later).
\section{System Architecture}
Below is the high-level architectural drawing of the system, illustrating the interactions between microservices.
\begin{figure}[H]
    \centering
    \includegraphics[width=0.8\textwidth]{architecture.png}
    \label{fig:arch}
\end{figure}

All microservices are containerized using Docker and are written in Python. Some of them use the Flask framework and some FastAPI.

\begin{table}[H]
    \centering
    \begin{tabular}{|l|l|l|}
        \hline
        \textbf{Microservice} & \textbf{Framework} & \textbf{Description} \\
        \hline
        Game history   & Flask   & Stores and retrieves past game data \\
        Game engine    & Flask   & Handles the game runtime \\
        Collection     & Flask   & Handles cards and users decks \\
        User Editor    & FastAPI & Manages user profiles and settings \\
        User Manager   & FastAPI & Manages user authentication and authorization \\
        Gateway        & FastAPI & Routes requests to appropriate microservices \\
        \hline
    \end{tabular}
\end{table}

\subsection{Design choices and microservices interactions}
\begin{itemize}
    \item \textbf{Centralized Authorization:} (draft)Every microservice connects to the User Manager to verify the token header (so that i can get dynamic username). User manager logs out users when they change username, this means that if we used decentralized auth, a logged-in user could see its old username using an old token, also, we don't have a way to force the log-out of an user with decentralized
    \item \textbf{User Information Update:} We've separated the user profile editing endpoints from the user manager to keep the user manager smaller and allow horizontal scalability. The User Editor then connects to a single endpoint of the User Manager that allows to update user information in the DB.
    \item \textbf{}
    \item \textbf{Separated in-game logic:} The Game Engine microservice handles the whole game logic, it
    \item \textbf{Hard-coded cards:} Since as cards we've choosen the classical French-suited deck, that we don't expect to change, we've decided to hard-code their behavior and their images in the microservices, without relying on databases.
\end{itemize}



% --- 1-3 Pages: User Stories (details later).
\section{User Stories}
\newcounter{usCounter}
\newcommand{\colorednum}[2]{%
    \stepcounter{usCounter}%
    \textcolor{#1}{\textbf{\theusCounter.}}
}

\begin{itemize}
    \item Account
    \begin{itemize}

        \item[\colorednum{red}{}] Create an account SO THAT I can participate to the game
        \begin{itemize}
            \item /users/register (Gateway $\rightarrow$ User Manager $\rightarrow$ MongoDB $\rightarrow$ User Manager $\rightarrow$ Gateway)
        \end{itemize}


        \item[\colorednum{red}{}] Login into the game SO THAT I can play a match
        \begin{itemize}
            \item /users/login (Gateway $\rightarrow$ User Manager $\rightarrow$ MongoDB $\rightarrow$ User Manager $\rightarrow$ Gateway)
        \end{itemize}


        \item[\colorednum{red}{}] Check/modify my profile SO THAT I can update my information
        \begin{itemize}
            \item /userseditor/modify/change-username (Gateway $\rightarrow$ User Editor $\rightarrow$ User Manager $\rightarrow$ MongoDB $\rightarrow$ User Manager $\rightarrow$ User Editor $\rightarrow$ Gateway)
            \item /userseditor/modify/change-password (Gateway $\rightarrow$ User Editor $\rightarrow$ User Manager $\rightarrow$ MongoDB $\rightarrow$ User Manager $\rightarrow$ User Editor $\rightarrow$ Gateway)
            \item /userseditor/modify/change-email (Gateway $\rightarrow$ User Editor $\rightarrow$ User Manager $\rightarrow$ MongoDB $\rightarrow$ User Manager $\rightarrow$ User Editor $\rightarrow$ Gateway)
        \end{itemize}
        	\textit{Note: Profile checking is handled via token validation or login response.}

        \item[\colorednum{yellow}{}] Be safe about my account data SO THAT nobody can steal/modify them
        \begin{itemize}
            \item Implemented via JWT Authentication, Password Hashing (Argon2) in User Manager, and HTTPS communication.
        \end{itemize}
    \end{itemize}
    \item Cards
    \begin{itemize}

        \item[\colorednum{red}{}] See the overall card collection SO THAT I can think of a strategy
        \begin{itemize}
            \item /collection/cards (Gateway $\rightarrow$ Collection $\rightarrow$ File System [cards.json] $\rightarrow$ Collection $\rightarrow$ Gateway)
        \end{itemize}


        \item[\colorednum{red}{}] View the details of a card SO THAT I can think of a strategy
        \begin{itemize}
            \item /collection/cards/{card\_id} (Gateway $\rightarrow$ Collection $\rightarrow$ File System [cards.json] $\rightarrow$ Collection $\rightarrow$ Gateway)
        \end{itemize}
    \end{itemize}
    \item Game
    \begin{itemize}

        \item[\colorednum{red}{}] Start a new game SO THAT I can play
        \begin{itemize}
            \item /game/match/join (Gateway $\rightarrow$ Game Engine $\rightarrow$ Matchmaking Queue $\rightarrow$ Game Engine $\rightarrow$ Gateway)
        \end{itemize}


        \item[\colorednum{red}{}] Select the subset of cards SO THAT I can start a match
        \begin{itemize}
            \item /game/deck/{game\_id} (Gateway $\rightarrow$ Game Engine $\rightarrow$ Game State $\rightarrow$ Gateway)
            \item Alternatively for deck creation: /collection/decks (Gateway $\rightarrow$ Collection $\rightarrow$ MongoDB $\rightarrow$ Collection $\rightarrow$ Gateway)
        \end{itemize}


        \item[\colorednum{red}{}] Select a card SO THAT I can play my turn
        \begin{itemize}
            \item /game/play/{game\_id} (Gateway $\rightarrow$ Game Engine $\rightarrow$ Game Logic $\rightarrow$ Game State $\rightarrow$ Gateway)
        \end{itemize}


        \item[\colorednum{red}{}] Know the score SO THAT I know if I am winning or losing
        \begin{itemize}
            \item /game/state/{game\_id} (Gateway $\rightarrow$ Game Engine $\rightarrow$ Game State $\rightarrow$ Gateway)
        \end{itemize}


        \item[\colorednum{red}{}] Know the turns SO THAT I can know how many rounds there are till the end
        \begin{itemize}
            \item /game/state/{game\_id} (Gateway $\rightarrow$ Game Engine $\rightarrow$ Game State $\rightarrow$ Gateway)
        \end{itemize}


        \item[\colorednum{red}{}] See the score SO THAT I know who is winning
        \begin{itemize}
            \item /game/state/{game\_id} (Gateway $\rightarrow$ Game Engine $\rightarrow$ Game State $\rightarrow$ Gateway)
        \end{itemize}


        \item[\colorednum{red}{}] Know who won the turn SO THAT I know the updated score
        \begin{itemize}
            \item /game/state/{game\_id} (Gateway $\rightarrow$ Game Engine $\rightarrow$ Game State $\rightarrow$ Gateway)
        \end{itemize}


        \item[\colorednum{red}{}] Know who won a match SO THAT I know the result of a match
        \begin{itemize}
            \item /game/state/{game\_id} (Gateway $\rightarrow$ Game Engine $\rightarrow$ Game State $\rightarrow$ Gateway)
        \end{itemize}

        \item[\colorednum{yellow}{}] Ensure that the rules are not violated SO THAT I can play a fair match
        \begin{itemize}
            \item N/A (Enforced by Logic): Handled internally by Game Engine during /game/play/{game\_id} execution.
            \item (draft) zero trust for game engine and collection,
        \end{itemize}
    \end{itemize}
    \item Others
    \begin{itemize}

        \item[\colorednum{red}{}] View the list of my old matches SO THAT I can see how I played
        \begin{itemize}
            \item /history/matches (Gateway $\rightarrow$ Game History $\rightarrow$ MongoDB $\rightarrow$ Game History $\rightarrow$ Gateway)
        \end{itemize}


        \item[\colorednum{red}{}] View the details of one of my matches SO THAT I can see how I played
        \begin{itemize}
            \item /history/matches (Gateway $\rightarrow$ Game History $\rightarrow$ MongoDB $\rightarrow$ Game History $\rightarrow$ Gateway)
        \end{itemize}
        	extit{Note: The list endpoint returns match details.}


        \item[\colorednum{red}{}] View the leaderboards SO THAT I know who are the best players
        \begin{itemize}
            \item /history/leaderboard (Gateway $\rightarrow$ Game History $\rightarrow$ MongoDB $\rightarrow$ Game History $\rightarrow$ Gateway)
        \end{itemize}

        \item[\colorednum{yellow}{}] Prevent people to tamper my old matches SO THAT I have them available
        \begin{itemize}
            \item N/A (Security): Implemented via internal service isolation (only Game Engine writes to History via RabbitMQ/Internal API) and Database access controls.
        \end{itemize}
    \end{itemize}
\end{itemize}


% --- 1-2 Pages: Game rules (details later).
\section{Rules of the Game}
\subsection{Card Game Rules}

\subsubsection{Deck Building Constraints}
Each player constructs a \textbf{personal deck} of exactly \textbf{9 cards}. The deck must adhere to the following constraints:

\begin{itemize}
    \item \textbf{Composition:} The deck must contain exactly \textbf{1 Joker} and \textbf{8 Suited Cards}.
    \item \textbf{Suit Distribution:} You must include exactly \textbf{2 cards} from each suit (\redheart, \reddiamond, \blackclub, \blackspade).
    \item \textbf{Cost Limit:} The combined point value of the two cards in any single suit \textbf{must not exceed 15}.
\end{itemize}

\textbf{Card Point Costs} \\
When calculating your deck limits, use the following costs (Note: Face cards have specific costs despite their combat strength):

\begin{table}[h!]
    \centering
    \begin{tabular}{lccc}
        \toprule
        \textbf{Card Type} & \textbf{Ranks} & \textbf{Cost per Card} & \textbf{Example Pair Limit} \\
        \midrule
        Numbers & 2 -- 10 & Face Value & $10 + 5 = 15$ \checkmark \\
        Ace & A & 7 & $A + 8 = 15$ \checkmark \\
        Jack & J & 11 & $J + 4 = 15$ \checkmark \\
        Queen & Q & 12 & $Q + 3 = 15$ \checkmark \\
        King & K & 13 & $K + 2 = 15$ \checkmark \\
        \bottomrule
    \end{tabular}
\end{table}

\subsubsection{Combat Hierarchy}
When two cards are played, the winner is decided by the following hierarchy:

\paragraph{1. The Combat Triangle}
The core mechanics function like Rock-Paper-Scissors:
\begin{itemize}
    \item \textbf{Numbers (2--10)} beat \textbf{Aces}.
    \item \textbf{Aces} beat \textbf{Face Cards (J, Q, K)}.
    \item \textbf{Face Cards (J, Q, K)} beat \textbf{Numbers}.
\end{itemize}

\paragraph{2. The Joker}
The \textbf{Joker} beats all other cards automatically. (If both players play a Joker, it is a tie).

\paragraph{3. Tie-Breakers}
If the combat rules above do not determine a clear winner (e.g., Number vs Number, or Face vs Face), compare the specific ranks:
\begin{enumerate}
    \item \textbf{Higher Rank Wins:} (e.g., 9 beats 6, King beats Jack).
    \item \textbf{Equal Rank $\rightarrow$ Suit Priority:} If ranks are identical (e.g., \redheart 9 vs \reddiamond 9), check suits:
    \[ \redheart > \reddiamond > \blackclub > \blackspade \]
    \item \textbf{Mirror Match:} If players play the \textit{exact same card} (Rank and Suit), both players win the round and gain a point.
\end{enumerate}

\subsubsection{Gameplay Flow}
\begin{enumerate}
    \item \textbf{Setup:} Both players shuffle their pre-built decks.
    \item \textbf{Initial Draw:} Each player draws \textbf{3 cards} to form their starting hand.
    \item \textbf{The Round:}
    \begin{itemize}
        \item \textbf{Draw Phase:} At the start of every turn (including Turn 1), players draw \textbf{1 card}.
        \item \textbf{Battle Phase:} Both players play one card \textbf{face down}, then reveal simultaneously.
        \item \textbf{Scoring:} Determine the winner based on the Combat Hierarchy. The winner earns \textbf{1 point}.
    \end{itemize}
    \item \textbf{Victory:} The game ends immediately when a player reaches \textbf{5 points}.
\end{enumerate}

\subsubsection{Example Match Log}
\textbf{Alice's Deck:} \redheart A, \redheart 7, \reddiamond A, \reddiamond 7, \blackclub K, \blackclub 2, \blackspade K, \blackspade 2, Joker.\\
\textbf{Bob's Deck:} \redheart 2, \redheart 3, \reddiamond 2, \reddiamond 3, \blackclub 3, \blackclub 4, \blackspade 2, \blackspade 3, Joker.

\vspace{1em}

\begin{table}[h!]
    \centering
    \small % Makes the table slightly smaller to fit nicely
    \begin{tabular}{ccclc}
        \toprule
        \textbf{Turn} & \textbf{Alice} & \textbf{Bob} & \textbf{Result Reasoning} & \textbf{Score (A-B)} \\
        \midrule
        1 & \redheart A & \blackclub K & \textbf{Ace beats Face} (Special Rule) & 1 -- 0 \\
        2 & \reddiamond A & \reddiamond 3 & \textbf{Number beats Ace} (Special Rule) & 1 -- 1 \\
        3 & \blackclub K & \blackclub 3 & \textbf{Face beats Number} (Special Rule) & 2 -- 1 \\
        4 & \blackspade K & \blackspade 2 & \textbf{Face beats Number} (Special Rule) & 3 -- 1 \\
        5 & \redheart 7 & \blackclub 4 & Both Numbers: $7 > 4$ & 3 -- 2 \\
        6 & \reddiamond 7 & \blackspade 3 & Both Numbers: $7 > 3$ & 3 -- 3 \\
        7 & \blackclub 2 & Joker & \textbf{Joker beats Everything} & 3 -- 4 \\
        8 & Joker & \redheart 2 & \textbf{Joker beats Everything} & 4 -- 4 \\
        9 & \redheart 7 & \blackspade 3 & Both Numbers: $7 > 3$ & \textbf{5 -- 4} \\
        \bottomrule
    \end{tabular}
\end{table}


% --- 1-2 Pages: Game flow (details later)
\section{API Calls of the Game Flow}
\subsection{Deck Building Phase}
We've separated the deck building from the game (More information in \hyperref[sec:cloud-decks]{Cloud Storage of Decks})
\begin{itemize}
    \item 
\end{itemize}
\subsection{Game Phase}
\begin{itemize}
    \item 
\end{itemize}

% --- 1-2 Pages: Testing (details later).
\section{Testing}

The project implements a testing strategy covering unit tests, integration tests, and performance tests to ensure system reliability, correctness, and scalability across all microservices.

\subsection{Unit Testing}
All tests are located in \texttt{/docs/tests} as requested.

Unit tests are executed using dedicated Dockerfile\_test files like we've seen in the lectures.\\
We made the following decisions for mocking.
\begin{itemize}
    \item For all microservices the databases are mocked using the mongomock library, installed in the Dockerfile\_test file.
    \item To mock the RabbitMQ for Game History we disabled the loop pinging the RabbitMQ microservice.
    This meant that we needed custom endpoints to add matches and users to the DB in the testing environment.
    \texttt{/addmatches} and \texttt{/addusernames}.
    \item In Game History we also needed to mock the function to get the usernames by ids of User Manager.
\end{itemize}

\textbf{Unit tests execution:}
\begin{itemize}
    \item \textbf{Collection}:\\
    \begin{lstlisting}[language=bash]
docker build -f collection/Dockerfile_test -t collection-test .
docker run -p 5000:5000 collection-test
newman run docs/tests/collection_ut.postman_collection.json --insecure
    \end{lstlisting}
    \item \textbf{Game History}:\\
    \begin{lstlisting}[language=bash]
docker build -f game_history/Dockerfile_test -t history-test .
docker run -p 5000:5000 history-test
newman run docs/tests/game_history_ut.postman_collection.json --insecure
    \end{lstlisting}
    \item \textbf{User Manager}:\\
    \begin{lstlisting}[language=bash]
docker build -f user-manager/Dockerfile_test -t user-manager-test .
docker run -p 5004:5000 user-manager-test
newman run docs/tests/user_manager_ut.postman_collection.json --insecure
    \end{lstlisting}
\end{itemize}

\subsection{Integration Testing}

\begin{itemize}
    \item \textbf{IT-001: Complete Game Workflow - Happy Path}: Tests end-to-end user journey from registration to game completion.
    \item \textbf{IT-002: Authentication \& Authorization}: Verifies login, token validation, and access control enforcement.
    \item \textbf{IT-003: Deck Validation}: Ensures deck building rules and constraints are correctly enforced.
    \item \textbf{IT-004: Game History \& Leaderboard}: Checks match history retrieval and leaderboard accuracy.
    \item \textbf{IT-005: Cross-Service Data Consistency}: Confirms data isolation and consistency across microservices.
    \item \textbf{IT-007: Error Handling \& Edge Cases}: Validates error responses and handling of invalid or edge-case inputs.
\end{itemize}

\textbf{Integration tests execution:}
\begin{itemize}
    \item \textbf{Docker Compose}:\\
    \begin{lstlisting}[language=bash]
# Ensure all services are running
cd src
docker compose up --build

# Run integration tests (from project root)
newman run docs/tests/integration.postman_collection.json --insecure
    \end{lstlisting}
\end{itemize}

Those tests can also 

\begin{center}
    {\Huge\color{red}\textbf{!}}
\end{center}
\textbf{Using Python Script:}
A complete game simulation can be executed programmatically:
\begin{lstlisting}[language=bash]
cd src
python test_match.py
\end{lstlisting}

This script performs:
\begin{enumerate}
    \item Registers two random users with unique credentials
    \item Creates valid decks for both players
    \item Initiates matchmaking and pairs the users
    \item Simulates a complete game with card plays
    \item Displays comprehensive game statistics including:
    \begin{itemize}
        \item Round-by-round results
        \item Final scores
        \item Winner determination
        \item Game duration
    \end{itemize}
\end{enumerate}

\subsection{Performance Testing with Locust}

Performance tests simulate realistic user load to measure system behavior under concurrent access and identify bottlenecks. 
The tests use Locust, a Python-based load testing framework.

\subsubsection{Test Scenarios}

The performance test suite simulates a complete user workflow representing realistic usage patterns:

\begin{enumerate}
    \item \textbf{User Registration}: Creates new accounts with random credentials
    \item \textbf{Authentication}: Performs login and JWT token generation
    \item \textbf{Deck Creation}: Builds valid 8-card decks following game rules
    \item \textbf{Matchmaking}: Joins queue and waits for opponent matching
    \item \textbf{Gameplay}: Simulates complete game sessions with:
    \begin{itemize}
        \item Deck selection for matched game
        \item Iterative card plays
        \item Hand retrieval between turns
        \item Game state validation
    \end{itemize}
    \item \textbf{History Access}: Queries match history and leaderboard data
\end{enumerate}

\subsubsection{User Types}

Three user types with different think times simulate varied usage patterns:

\begin{itemize}
    \item \textbf{QuickUser}: 1-3 second wait time between actions (rapid gameplay)
    \item \textbf{NormalUser}: 3-7 second wait time (typical gameplay)
    \item \textbf{SlowUser}: 5-15 second wait time (casual gameplay)
\end{itemize}

\subsubsection{Implementation Details}

The Locust test implementation (\texttt{docs/locustfile.py}) features:

\begin{itemize}
    \item \textbf{Sequential Task Execution}: \texttt{GameUserFlow} class orchestrates the complete workflow
    \item \textbf{Session Management}: Each user maintains state across requests:
    \begin{itemize}
        \item Username with random suffix (e.g., \texttt{loadtest\_user\_12345})
        \item JWT token for authenticated requests
        \item Active game ID during matches
        \item Selected deck slot
    \end{itemize}
    \item \textbf{SSL Configuration}: Disables certificate verification for self-signed certificates
    \item \textbf{Error Handling}: Graceful handling of concurrent access scenarios:
    \begin{itemize}
        \item 400 responses during registration marked as success, duplicate usernames expected and should not impact on the test results
        \item 401 responses during opponent's turn marked as success, they indicate it's not the user's turn and should not impact on the test results
        \item Failed operations are logged but don't interrupt test flow
    \end{itemize}
\end{itemize}

\subsubsection{Test Execution}

\textbf{Setup:}
\begin{lstlisting}[language=bash]
# Install Locust
pip install locust

# Ensure all services are running
cd src
docker compose up -d
\end{lstlisting}

\textbf{Running Tests:}
\begin{lstlisting}[language=bash]
# Start Locust web interface
cd docs
locust

# Access web UI at http://localhost:8089
\end{lstlisting}

\textbf{Configuration:}
\begin{enumerate}
    \item Set number of users (e.g., 50 concurrent users)
    \item Set spawn rate (e.g., 4 users/second)
    \item Set host: \texttt{https://localhost:8443}
    \item Click "Start" to begin test
\end{enumerate}

\subsubsection{Metrics and Analysis}

Locust provides real-time metrics during test execution:

\begin{itemize}
    \item \textbf{Request Statistics}:
    \begin{itemize}
        \item Requests per second (RPS) by endpoint
        \item Response time percentiles (50th, 95th, 99th)
        \item Failure rates and error types
        \item Average response sizes
    \end{itemize}
    
    \item \textbf{Endpoints}:
    \begin{itemize}
        \item \texttt{/users/register}: user creation
        \item \texttt{/users/login}: authentication
        \item \texttt{/collection/decks}: deck management operations
        \item \texttt{/game/match/join}: matchmaking queue
        \item \texttt{/game/hand}: cards in hand
        \item \texttt{/game/play}: card play action
        \item \texttt{/history/matches}: user's matches history
        \item \texttt{/history/leaderboard}: leaderboard
    \end{itemize}

    % Analisi da terminare
\end{itemize}

\section{Security}

\subsection{Security – Data}

\subsubsection{Input Sanitization}
\textbf{Target:} User email addresses (Registration/Modification).

\textbf{Threats Mitigated:} NoSQL injection, XSS, Duplicate accounts.

\textbf{Implementation:}
\begin{itemize}
    \item \textbf{Type Validation:} Pydantic models enforce strict typing (e.g., \texttt{email: str}).
    \item \textbf{Duplicate Prevention:} SHA-256 hashing of emails for secure, consistent search keys without exposing raw data.
    \item \textbf{Query Safety:} Explicit type casting of query parameters (e.g., \texttt{page=int}) to prevent injection.
\end{itemize}

\subsubsection{Data Encryption at Rest}
\textbf{Scope:} Sensitive user data (Email, Username) in \texttt{user-db}.

\textbf{Mechanism:}
\begin{itemize}
    \item \textbf{Algorithm:} Fernet (Symmetric encryption).
    \item \textbf{Key Management:} 32-byte cryptographically secure key, stored as Docker secret (\texttt{/run/secrets/user\_db\_encryption\_secret\_key}).
    \item \textbf{Process:} 
    \begin{itemize}
        \item \textbf{Write:} Encrypt data before insertion.
        \item \textbf{Read:} Decrypt data on retrieval.
        \item \textbf{Search:} Use hashed values for lookups to avoid decrypting entire collections.
    \end{itemize}
\end{itemize}

\subsection{Security – Authentication and Authorization}

\subsubsection{Architecture}
\textbf{Model:} Centralized Authentication via \texttt{user-manager}.

\textbf{Flow:}
\begin{enumerate}
    \item \textbf{Login:} Client credentials $\rightarrow$ API Gateway $\rightarrow$ \texttt{user-manager} $\rightarrow$ JWT issued.
    \item \textbf{Access:} Client sends JWT $\rightarrow$ Microservice $\rightarrow$ Validate with \texttt{user-manager}.
\end{enumerate}

\subsubsection{JWT Implementation}
\textbf{Configuration:}
\begin{itemize}
    \item \textbf{Algorithm:} HS256.
    \item \textbf{Secret Storage:} Docker secret (\texttt{/run/secrets/jwt\_secret\_key}), accessible only to \texttt{user-manager}.
    \item \textbf{Payload:}
    \begin{itemize}
        \item \texttt{sub}: Username (Subject).
        \item \texttt{id}: Database ObjectId.
        \item \texttt{exp}: Expiration timestamp (30 minutes).
    \end{itemize}
\end{itemize}

\subsubsection{Token Lifecycle}
\begin{itemize}
    \item \textbf{Validation:} \texttt{pyJWT} library automatically verifies signature and expiration.
    \item \textbf{Expiration Policy:} For simplicity and security, we do not implement refresh tokens. Users must re-authenticate after 30 minutes of inactivity. This reduces the attack surface by limiting token lifetime.
    \item \textbf{Error Handling:} Expired/Invalid tokens return HTTP 401, triggering client-side logout.
\end{itemize}

\newpage
\subsection{Security – Analyses}

\subsubsection{Static Analysis with Bandit}

Bandit was used to perform static application security testing on the Python codebase.

\textbf{Command executed:}
\begin{lstlisting}
bandit -r src/
\end{lstlisting}
\begin{figure}[h]
\centering
\includegraphics[width=0.4\textwidth]{images/bandit.png}
\end{figure}

% Da aggiungre screenshot del report di Bandit
% \begin{figure}[h]
% \centering
% \includegraphics[width=0.95\textwidth]{bandit_report.png}
% \caption{Bandit static analysis final summary}
% \end{figure}

\subsubsection{Dependency Vulnerability Scanning}

\paragraph{Pip Audit:}
Executed as a github action.
It found vulnerabilities before but not after merging the Dependabot pull request.

\paragraph{Dependabot (GitHub):}
Enabled on the repository to automatically scan for vulnerable dependencies.

\begin{itemize}
    \item Updates python-multipart from 0.0.9 to 0.0.18
    \item Updates requests from 2.31.0 to 2.32.4
    \item Updates cryptography from 42.0.7 to 44.0.1
\end{itemize}
This also solved issues raised by pip-audit

\subsubsection{Container Image Vulnerability Analysis}

\paragraph{Docker Scout:}

\textbf{Usage:} all container images where scanend for vulnerabilities using Docker Scout directly from Docker Desktop.

\textbf{Mitigation Actions:}
\begin{itemize}
    \item \textbf{RabbitMQ:} 2 critical, 9 high-severity vulnerabilities because of the golang and stdlib versions in the linux image used by 3-management. Solved upgrading to \texttt{rabbitmq:4.2.1-alpine}.
    \item \textbf{MongoDB:}
    This was more of an issue, 
    \begin{itemize}
        \item The mongo:latest image we used in development had 6 high-severity vulnerabilities.
        \item All the versions of the official mongo image have either a lot of vulnerabilities or run only on windows servers (windows OS images).
        \item "latest" images are not an option as a final shipment version, this would mean the images possibly breaking for an update.
        \item All the companies we've found providing mongo linux images moved to a paid model, providing only "latest" versions for free for development environments.
        \item The only option we had were old legacy images made by those companies in the past.
        Fortunately we found \textit{circleci/mongo:4.0-xenial-ram},
        a 4 years old image that relied on a really light-weight ubuntu version with 0 vulnerabilities.
        \item This is definitely not a future-proof solution. The best path forward would probably be to build a custom image where we install mongoDB on top of a lightweight alpine or debian image.
    \end{itemize}
    \item \textbf{JWT:} The User Manager microservice used an abandoned library to handle the JWT called python-jose with a high severity vulnerability. Fortunately there was a new and updated library that had the same functions, called PyJWT.
    \item FastAPI 0.111.0 has critical vulnerabilities caused by a starlette dependency. Solved by upgrading to 0.124.4.
    \item python:3.12.3-slim has critical vulnerabilities caused by the debian image used. Solved by upgrading to 3.12.12-slim.
\end{itemize}

% Da aggiungere screenshot del report di Docker Scout
% \begin{figure}[h]
% \centering
% \includegraphics[width=0.95\textwidth]{docker_scout_dashboard.png}
% \caption{Docker Scout vulnerability dashboard}
% \end{figure}

\paragraph{Trivy:}

\textbf{Command executed:}
\begin{lstlisting}
trivy image --severity HIGH,CRITICAL <image_name>
\end{lstlisting}

\textbf{Findings:}
\begin{itemize}
    \item \textbf{Exposed Secrets:} Trivy detected hardcoded keys in repository
    \item \textbf{Academic Context:} As agreed with the Professor, keys are intentionally included since it's a project with educational purposes.
    \item \textbf{Production Mitigation:} In a real deployment, keys would be:
    \begin{itemize}
        \item Generated dynamically per environment or, even better, we could've used certificates provided by a CA.
        \item Stored in secure secret management systems
        \item Never committed
    \end{itemize}
    The same principles would apply for databases passwords hardcoded in the docker compose file.
\end{itemize}


\subsection{Security – Threat Model}

\subsubsection{Architecture Overview}
The application follows a microservices architecture with a centralized API Gateway acting as the single entry point. All inter-service communication is secured via TLS.

\begin{itemize}
    \item \textbf{External Entry Point:} API Gateway (exposed on port 8443).
    \item \textbf{Internal Services:} User Manager, User Editor, Game Engine, Game History, Collection.
    \item \textbf{Data Stores:} Dedicated MongoDB instances for Users, History, and Decks.
    \item \textbf{Messaging:} RabbitMQ for asynchronous communication, configured to reject non-TLS connections.
\end{itemize}

\subsubsection{Assets Identification}
\begin{itemize}
    \item \textbf{User Credentials:} Passwords (hashed) and Emails (encrypted).
    \item \textbf{Session Tokens:} JWTs used for authentication.
    \item \textbf{Game Data:} Match history and player moves.
    \item \textbf{Infrastructure Secrets:} TLS certificates and private keys managed via Docker Secrets.
\end{itemize}

\subsubsection{STRIDE Analysis}

\paragraph{Spoofing}
\begin{itemize}
    \item \textbf{Threat:} An attacker impersonating a legitimate user.
    \item \textbf{Mitigation:} Strong authentication via \texttt{user-manager}, JWT validation on every request, and TLS for all connections to prevent credential interception.
    \item \textbf{Threat:} A rogue service impersonating a legitimate microservice.
    \item \textbf{Mitigation:} Internal TLS usage with certificates managed via Docker Secrets ensures encrypted communication channels.
\end{itemize}

\paragraph{Tampering}
\begin{itemize}
    \item \textbf{Threat:} Modification of game results or user data in transit.
    \item \textbf{Mitigation:} End-to-end TLS encryption prevents Man-in-the-Middle (MitM) attacks.
    \item \textbf{Threat:} Modification of data at rest.
    \item \textbf{Mitigation:} Critical user data (emails) is encrypted at rest. Database access is restricted to specific microservices.
\end{itemize}

\paragraph{Repudiation}
\begin{itemize}
    \item \textbf{Threat:} A user denying they performed an action (e.g., a game move).
    \item \textbf{Mitigation:} The \texttt{game\_history} service logs all match outcomes. Actions are authenticated via JWT, linking them irrefutably to a user ID.
\end{itemize}

\paragraph{Information Disclosure}
\begin{itemize}
    \item \textbf{Threat:} Leaking sensitive user information.
    \item \textbf{Mitigation:} Data minimization (only necessary data returned), encryption of sensitive fields in DB, and strict API Gateway routing preventing direct access to backend services.
\end{itemize}

\paragraph{Denial of Service (DoS)}
\begin{itemize}
    \item \textbf{Threat:} Overwhelming the system with requests.
    \item \textbf{Mitigation:} The API Gateway acts as a buffer. RabbitMQ decouples heavy processing (like history logging) from the critical path, preventing cascading failures.
\end{itemize}

\paragraph{Elevation of Privilege}
\begin{itemize}
    \item \textbf{Threat:} A regular user accessing administrative functions.
    \item \textbf{Mitigation:} Centralized authorization logic in \texttt{user-manager}. The API Gateway enforces route restrictions.
\end{itemize}



% --- 1-3 Pages: Threat model (details later).
\input{sections/threat_model.tex}

% --- 1-3 Pages: Use of Generative AI (details later).
\section{Use of Generative AI}
During the project development we've made extensive use of various AI models, experimenting with them and learning how to use those tools as efficiently as possible.
We've mainly taken advantage of what was given to us for free as students: GitHub Copilot Pro and Gemini Pro.
We've used them in the following two areas.
\begin{itemize}
    \item Researching informations and tools: this has gotten a lot better in the past years, with models combining their knowledge with informations found on the web, providing extensive explanations and sources, while almost never allucinating. Researching using LLMs has definitely almost replaced our reliance on classical search engines.
    \item Writing Code.
    To which we've made the following considerations.
    \begin{itemize}
        \item The non-premium AI models of Copilot Pro (GPT-4.1, GPT-4o, GPT-5 mini, Grok Code Fast 1) were pretty bad and mostly generated slop we lost time in reviewing and then discarding. This was true even in writing something as simple as latex code, e.g. GPT-4.1 had a hard time understanding what an hyperref was.
        \item We tried various AI coding VSCode extensions: Github Copilot, Roo Code, Gemini Code Assist etc. We found the Copilot extension as the most robust implementation of agentic behavior for vscode, and, together with the free premium requests, we ended up using it almost exclusively.
        \item We found Claude Sonnet 4.5 as the best premium model for one-shot code generation, it has been useful to generate some initial drafts for the microservices and to re-generate whole functions.
        \item We found the Copilot Gemini 3.0 Agent behaving in a more \textit{human} way, trying to get the best result while doing as least work as possible. E.g. for big changes in a single file that meant almost a whole refactoring, it often tried to delete the whole file and start from scratch.
        \item The inline suggestions were a dividing topic, while some of us found them useful and kept them on, some others found them distracting and fatigue-inducing, where they suggested mostly wrong code while constantly drawing the attention of the user.
        \item Even if ChatGPT models are still the most popular ones, we found them working on par, if not worse, than the others, hence we made less extensive use of them. This resonates with the recent "Code Red" raised by OpenAI after the release of Gemini 3.0.
    \end{itemize}
    
\end{itemize}
\subsection{Final Remarks}
We found experienced and smart human beings better in any way but speed compared to the most powerful AI models that are publicly available.
At the moment LLMs can be a useful tool for those people to work faster, but definitely not as a replacement of them.\\
Reasoning models behaved way better,
showing the robustness of more rigid and iterative workflows, applied to the more efficient and smaller MoE (Mixture of Experts) models, compared to one-shot larger models.
This resonates with what is shown in the course for cloud services, where decomposing complex tasks (monoliths) into smaller, more easily verifiable tasks (microservices), often yields better results.

% --- X Pages: Additional Features (details later).
\section{Additional Features}
%3 user stories verdi, client, salvataggio deck in collection, uso di rabbitmq come proof of concept

\subsection{Green User Stories}
\label{sec:green-user-stories}

The project implements two green user stories that enhance the player's gameplay experience by providing essential game state information.

\subsubsection{Viewing Cards in Hand}
\begin{itemize}
    \setcounter{usCounter}{21}
    \item[\colorednum{green}{2}] who's turn it is SO THAT I know if i can play or not
    \begin{itemize}
        \item \texttt{GET /state/\{game\_id\}} (Gateway $\rightarrow$ Game Engine $\rightarrow$ Gateway)
        \item Response: current round state (empty, one card played, both cards played)
    \end{itemize}
    \setcounter{usCounter}{23}
    \item[\colorednum{green}{1}] To be able to see the cards in my hand	SO THAT	I know that card I can play next

    \begin{itemize}
        \item \texttt{GET /hand/\{game\_id\}} (Gateway $\rightarrow$ Game Engine $\rightarrow$ Gateway)
        \item Returns: JSON array of card objects (\texttt{value}, \texttt{suit})
    \end{itemize}

\end{itemize}

\subsection{Cloud Storage of Decks}
\label{sec:cloud-decks}
To allow players to keep their decks stored in the cloud, we modified the way they handle decks.\\
Users must first create the decks connecting to the endpoint \texttt{collection/decks}. They have 5 slots (5 possible decks).\\
Then that deck will be stored on the decks database.\\
Users will then be able to choose one of their decks when starting a game with the endpoint \texttt{game/match/join}.

\subsection{Client}
The project includes a Python-based Command Line Interface (CLI) client located in the \texttt{/client} directory. This client serves as the frontend for the game, communicating with the backend services via the API Gateway. It uses the \texttt{rich} and \texttt{questionary} libraries to provide an interactive terminal experience.

To run the client using Docker, execute the following command (Linux):
\begin{lstlisting}
cd client/
docker build -t ase-client . --no-cache
docker run -it --rm \
  --add-host=host.docker.internal:host-gateway \
  -e API_GATEWAY_URL="https://host.docker.internal:8443/" \
  -e GATEWAY_CERT_PATH="./gateway_cert.pem" \
  ase-client
\end{lstlisting}
Ensure that the backend services are running before starting the client.

\subsubsection{User Authentication}
\begin{itemize}
    \item \textbf{Login/Register:} Users are prompted to authenticate via the API Gateway and Authentication microservice upon startup.
    \item \textbf{Token Management:} The client locally stores the username and JWT token. The token is included in the headers of every request (defined in \textbf{$``$apicalls.py$``$}).
    \item \textbf{Prerequisite:} Users must navigate to \textbf{$``$Decks$``$} to create at least one deck before playing.
\end{itemize}

\subsubsection{Card Collection and Deck Management}
\begin{itemize}
    \item \textbf{Capabilities:} Users can view the collection, create new decks (Deck Page), view existing decks, and delete decks (View Deck).
    \item \textbf{Visuals:} Selecting a card in the local version opens its image in a new window (not in Containerized version).
    \item \textbf{Deck Creation:} Simplified logic grants the maximum value for every suit during creation.
\end{itemize}

\subsubsection{Game Engine Interaction}
\begin{itemize}
    \item \textbf{Testing Setup:} Requires two terminal instances logged in with different user accounts.
    \item \textbf{Matchmaking:} Select \textbf{$``$Play a Match$``$} to enter the queue and await an opponent.
    \item \textbf{Gameplay:} The interface displays real-time game state (hand, scores, opponent name) and enables interactive card play per turn.
    \item \textbf{Conclusion:} Displays final results and automatically updates the user's game history.
\end{itemize}

\subsubsection{Game History}
\begin{itemize}
    \item \textbf{Access:} Located within the \textbf{$``$Leaderboard$``$} section of the main menu.
    \item \textbf{Options:} Users can toggle between the full Global Leaderboard and their Personal Match History.
    \item \textbf{Navigation:} The leaderboard includes a pagination system to browse all players.
\end{itemize}

\subsection{Endpoint-based Service Interaction Smell - Proof of Concept} %RabbitMQ
\label{sec:rabbitmq}
The project requirements asked to make up for the \textbf{Wobbly service interaction} smell. We solved it mainly with timeouts as suggested.\\
Regarding the \textbf{Endpoint-based service interaction} smell, the requirements said to ignore it since it would've been difficult to solve.\\
We decided to partially solve it anyway in a single interaction between the \textbf{Game History} and \textbf{Game Engine} microservices as a proof-of-concept.\\
In this interaction we replaced the timeout with RabbitMQ, a Message Broker, that solved both the \textbf{Wobbly service interaction} and the \textbf{Endpoint-based service interaction} smells.
This meant a new docker container for RrabbitMQ and running code to fill up the RabbitMQ queue in Game Engine and to ping it for new data in Game History.
\subsection{Test-match}
As already said in \hyperref[sec:test-match]{Test Match} we made a simple script to test the system. You can find it in \texttt{/docs/tests/test\_match.py}.

\subsection{pip-audit Actions}
Together with the actions for unit tests and integration tests, we implemented a GitHub Action that runs \texttt{pip-audit} to automatically check for known vulnerabilities in our Python dependencies.






% --- Section 4: Build & Run ---
\section{Build and Run Instructions}

This project is a multi-service card game platform designed for two players. It features authentication, deck building, a match-simulation engine, and history tracking. All services are containerized and orchestrated via Docker Compose.

\subsection{Prerequisites}
Before starting, ensure your environment meets the following requirements:
\begin{itemize}
    \item \textbf{Docker \& Docker Compose} (Required for orchestration).
    \item \textbf{Python 3.10+} (Optional, only required for local non-containerized development).
\end{itemize}

\subsection{Quick Start Guide}

\begin{enumerate}
    \item \textbf{Clone the repository}
    \begin{verbatim}
git clone https://github.com/ashleyspagnoli/ASE_project.git
cd ASE_project/src
    \end{verbatim}

    \item \textbf{Build and launch services}
    Run the following command to build the images and start the containers:
    \begin{verbatim}
docker compose up --build
    \end{verbatim}

    \item \textbf{Verify Service Status}
    Once the containers are running, the architecture exposes the following endpoints:

    \begin{table}[h!]
        \centering
        \begin{tabular}{llc}
            \toprule
            \textbf{Service Name} & \textbf{Responsibility} & \textbf{Local URL} \\
            \midrule
            User Manager & Authentication \& JWT & \texttt{https://localhost:5004} \\
            Collection & Deck Management & \texttt{http://localhost:5003} \\
            Game Engine & Core Logic \& Matchmaking & \texttt{http://localhost:5001} \\
            Game History & Match Logging & \texttt{http://localhost:5002} \\
            \bottomrule
        \end{tabular}
    \end{table}
\end{enumerate}

\subsection{Development \& Testing}

\subsubsection{Environment Configuration}
Each microservice is configured via environment variables. For specific configuration keys, refer to the \texttt{Dockerfile} and \texttt{requirements.txt} located in each service's directory.

\subsubsection{Testing the Workflow}
A Postman collection is provided for end-to-end testing. Import \texttt{game\_workflow.postman\_collection.json} into Postman to simulate a full lifecycle:
\begin{itemize}
    \item User Registration and Login (Token generation).
    \item Deck creation and validation.
    \item Matchmaking and gameplay simulation.
\end{itemize}

\subsubsection{Key API Endpoints}
\begin{description}
    \item[Authentication] \hfill \\
    \begin{itemize}
        \item \texttt{/users/register}: user registration [POST]
        \item \texttt{/users/login}: user login [POST]
        \item \texttt{/users/validate-token}: internal JWT token validation [GET]
    \end{itemize}
    
    \item[Deck Management] \hfill \\
    \begin{itemize}
        \item \texttt{/collection/cards}: get the collection of cards [GET]
        \item \texttt{/collection/decks}: create a new deck [POST]
    \end{itemize}
    
    \item[Game Engine] \hfill \\
    \begin{itemize}
        \item \texttt{/game/connect}: connect to start playing the game [POST]
        \item \texttt{/game/matchmake}: manual request to start the match [POST] (TO REMOVE)
        \item \texttt{/game/play/\{game\_id\}}: play a card [POST]
        \item \texttt{/game/state/\{game\_id\}}: get the state of the game [GET]
    \end{itemize}
    \item[Game History] \hfill \\
    \begin{itemize}
        \item \texttt{/leaderboard}: get the whole leaderboard [GET]
        \item \texttt{/matches}: get the history of your played matches [GET]
        \item \texttt{/addmatch}: memorize a new match [POST]
    \end{itemize}
    
\end{description}

\subsection{Maintenance}
To stop the application and remove containers/networks, run:
\begin{verbatim}
docker compose down
\end{verbatim}


\iffalse
% --- Section 5: Play Instructions (API) ---
\section{Instructions to Play a Match}
Below is the precise sequence of API requests required to complete a full match loop.

\subsection{1. Create a New Game}
\textbf{Endpoint:} \texttt{POST /api/game/new} \\
\textbf{Description:} Initializes a new game session.

\begin{lstlisting}
curl -X POST http://localhost:8080/api/game/new \
     -H "Content-Type: application/json" \
     -d '{"player_count": 2}'
\end{lstlisting}

\subsection{2. Join Game}
\textbf{Endpoint:} \texttt{POST /api/game/\{gameId\}/join}

\begin{lstlisting}
curl -X POST http://localhost:8080/api/game/101/join \
     -d '{"player_name": "Alice"}'
\end{lstlisting}

\subsection{3. Play a Card}
\textbf{Endpoint:} \texttt{PUT /api/game/\{gameId\}/play}

\begin{lstlisting}
curl -X PUT http://localhost:8080/api/game/101/play \
     -d '{"card": "Ace of Spades"}'
\end{lstlisting} 


% --- Section 6: Testing ---
\section{Running Tests}

The project includes a comprehensive test suite located in the project root under the filename \texttt{game\_workflow.postman\_collection.json}. There are two methods to execute these tests.

\subsection{Method 1: Using the Postman CLI (Recommended)}
For automated testing or terminal-based execution, ensure you have the Postman CLI utility installed. Navigate to the project root and execute the following command:

\begin{verbatim}
postman collection run game_workflow.postman_collection.json
\end{verbatim}

\noindent
If you are using \texttt{newman} (the open-source runner), the command is:

\begin{verbatim}
newman run game_workflow.postman_collection.json
\end{verbatim}

\subsection{Method 2: Using the Postman Desktop App}
If you prefer a visual interface or need to debug specific requests, you can run the tests manually by importing them in the Postman Desktop App.
\fi


%RabbitMQ
\end{document}


#Collection: usa lazy loading del DB
#Salvataggio decks -> additional feature
#Di agli altri della cosa che non facendo flush non stampava gli errori