\section{System Architecture}
\subsection{Architecture schema and main microservices}
Below is the high-level architectural drawing of the system, illustrating the interactions between microservices.
\begin{figure}[H]
    \centering
    \includegraphics[width=0.8\textwidth]{architecture.png}
    \label{fig:arch}
\end{figure}

All microservices are containerized using Docker and are written in Python.
Some of them use the Flask framework and some FastAPI.\\
Here is a list of the main microservices of the system:
\begin{table}[H]
    \centering
    \begin{tabular}{|l|l|l|}
        \hline
        \textbf{Microservice} & \textbf{Framework} & \textbf{Description} \\
        \hline
        Game history   & Flask   & Stores and retrieves past game data \\
        Game engine    & Flask   & Handles the game runtime \\
        Collection     & Flask   & Handles cards and users decks \\
        User Editor    & FastAPI & Manages user profiles and settings \\
        User Manager   & FastAPI & Manages user authentication and authorization \\
        Gateway        & FastAPI & Routes requests to appropriate microservices \\
        \hline
    \end{tabular}
\end{table}

\subsection{Design choices and microservices interactions}
\begin{itemize}
    \item \textbf{Centralized Authorization:}
    every microservice connects to the User Manager to verify the token header, this is for two main reasons:
    \begin{itemize}
        \item The users can change their usernames that are stored in the User-DB, this is a way for microservices to verify they have the updated username.
        \item In the future we may want to add token revocation, having a single microservice handle that would make it easier.
    \end{itemize}
    \item \textbf{Shared persistance smell solutions:}
    \begin{itemize}
        \item We connected a single microservice to each database (Games, Decks and Users).
        \item Since as cards we've choosen the classical French-suited deck, that we don't
        expect to change, we've decided to hard-code their behavior and their images
        in the microservices, without relying on databases.
    \end{itemize}
    \item \textbf{User Editor:}
    We've separated the user profile editing endpoints from the user manager to keep the user manager smaller
    and allow horizontal scalability.
    The User Editor then connects to dedicated endpoints of the User Manager that allow it to update user information in the DB User-DB.
    \item \textbf{Cards and decks:}
    The Collection microservice allows the users to see all the cards, their images and to \hyperref[sec:cloud-decks]{handle their decks}. 
    \item \textbf{Separated in-game logic:}
    The Game Engine microservice, as the name suggest, handles the whole game logic,
    while delegating to Game History and Collection the handling of old games and decks.
    
\end{itemize}

