\section{System Architecture}
Below is the high-level architectural drawing of the system, illustrating the interactions between microservices.
\begin{figure}[H]
    \centering
    \includegraphics[width=0.8\textwidth]{architecture.png}
    \label{fig:arch}
\end{figure}

All microservices are containerized using Docker and are written in Python. Some of them use the Flask framework and some FastAPI.

\begin{table}[H]
    \centering
    \begin{tabular}{|l|l|l|}
        \hline
        \textbf{Microservice} & \textbf{Framework} & \textbf{Description} \\
        \hline
        Game history   & Flask   & Stores and retrieves past game data \\
        Game engine    & Flask   & Handles the game runtime \\
        Collection     & Flask   & Handles cards and users decks \\
        User Editor    & FastAPI & Manages user profiles and settings \\
        User Manager   & FastAPI & Manages user authentication and authorization \\
        Gateway        & FastAPI & Routes requests to appropriate microservices \\
        \hline
    \end{tabular}
\end{table}

\subsection{Design choices and microservices interactions}
\begin{itemize}
    \item \textbf{Centralized Authorization:} (draft)Every microservice connects to the User Manager to verify the token header (so that i can get dynamic username). User manager logs out users when they change username, this means that if we used decentralized auth, a logged-in user could see its old username using an old token, also, we don't have a way to force the log-out of an user with decentralized
    \item \textbf{User Information Update:} We've separated the user profile editing endpoints from the user manager to keep the user manager smaller and allow horizontal scalability. The User Editor then connects to a single endpoint of the User Manager that allows to update user information in the DB.
    \item \textbf{}
    \item \textbf{Separated in-game logic:} The Game Engine microservice handles the whole game logic, it
    \item \textbf{Hard-coded cards:} Since as cards we've choosen the classical French-suited deck, that we don't expect to change, we've decided to hard-code their behavior and their images in the microservices, without relying on databases.
\end{itemize}

