\section{Testing}

The project implements a testing strategy covering unit tests, integration tests, and performance tests to ensure system reliability, correctness, and scalability across all microservices.

\subsection{Unit Testing}
All tests are located in \texttt{/docs/tests} as requested.

Unit tests are executed using dedicated Dockerfile\_test files like we've seen in the lectures.\\
We made the following decisions for mocking.
\begin{itemize}
    \item For all microservices the databases are mocked using the mongomock library, installed in the Dockerfile\_test file.
    \item To mock the RabbitMQ for Game History we disabled the loop pinging the RabbitMQ microservice.
    This meant that we needed custom endpoints to add matches and users to the DB in the testing environment.
    \texttt{/addmatches} and \texttt{/addusernames}.
    \item In Game History we also needed to mock the function to get the usernames by ids of User Manager.
\end{itemize}

\textbf{Unit tests execution:}
\begin{itemize}
    \item \textbf{Collection}:\\
    \begin{lstlisting}[language=bash]
docker build -f collection/Dockerfile_test -t collection-test .
docker run -p 5000:5000 collection-test
newman run docs/tests/collection_ut.postman_collection.json --insecure
    \end{lstlisting}
    \item \textbf{Game History}:\\
    \begin{lstlisting}[language=bash]
docker build -f game_history/Dockerfile_test -t history-test .
docker run -p 5000:5000 history-test
newman run docs/tests/game_history_ut.postman_collection.json --insecure
    \end{lstlisting}
    \item \textbf{User Manager}:\\
    \begin{lstlisting}[language=bash]
docker build -f user-manager/Dockerfile_test -t user-manager-test .
docker run -p 5004:5000 user-manager-test
newman run docs/tests/user_manager_ut.postman_collection.json --insecure
    \end{lstlisting}
\end{itemize}

\subsection{Integration Testing}

The integration test suite comprises 11 test categories with 74 total requests and 93 assertions, providing comprehensive coverage of all microservices interactions.

\begin{itemize}
    \item \textbf{IT-001: Complete Game Workflow - Happy Path}: Tests end-to-end user journey from registration to game completion.
    \item \textbf{IT-002: Authentication \& Authorization}: Verifies login, token validation, and access control enforcement.
    \item \textbf{IT-003: Deck Validation}: Ensures deck building rules and constraints are correctly enforced.
    \item \textbf{IT-004: Game History \& Leaderboard}: Checks match history retrieval and leaderboard accuracy.
    \item \textbf{IT-005: Cross-Service Data Consistency}: Confirms data isolation and consistency across microservices.
    \item \textbf{IT-006: Advanced Game Scenarios}: Tests gameplay edge cases including invalid card plays and game state verification.
    \item \textbf{IT-007: Error Handling \& Edge Cases}: Validates error responses and handling of invalid or edge-case inputs.
    \item \textbf{IT-008: User Editor Integration}: Verifies user profile management, username changes, and data consistency.
    \item \textbf{IT-009: Complete Game Playthrough}: Tests all game engine endpoints through a complete match workflow including matchmaking, hand retrieval, card playing, and game state management.
    \item \textbf{IT-010: Complete Game Until Winner}: Simulates a complete 4-round game from start to finish and verifies match history persistence.
    \item \textbf{IT-011: Leaderboard and Statistics}: Validates leaderboard functionality, pagination, match history retrieval, and statistics tracking.
\end{itemize}

\textbf{Integration tests execution:}
\begin{itemize}
    \item \textbf{Docker Compose}:\\
    \begin{lstlisting}[language=bash]
# Ensure all services are running
cd src
docker compose up --build

# Run integration tests (from project root)
newman run docs/tests/integration.postman_collection.json --insecure
    \end{lstlisting}
\end{itemize}

Those tests are also automatically executed at each push on GitHub with GitHub Actions.\\
The unit tests will start after the pip-audit actions doesn't find vulnerabilities.\\
The integration test will start after the unit tests finish successfully.

\subsection{Performance Testing with Locust}

Performance tests were conducted using Locust (\texttt{docs/locustfile.py}) to simulate realistic user workflows and measure system behavior under load.

\textbf{Tested Workflow:}
\begin{itemize}
    \item User registration and login
    \item Deck creation
    \item Matchmaking and gameplay (card play, hand retrieval)
    \item History and leaderboard queries
\end{itemize}

Three user profiles (quick, normal, slow) with different wait times were used to mimic varied usage patterns.

\textbf{Execution:}
\begin{itemize}
    \item 100 concurrent users, spawn rate 20 users/sec
    \item No errors observed during the test
    \item The slowest part of the system was the one handled by User Manager. This was probably because of the Argon2 hashing algorithm used for passwords, which is intentionally slow to prevent brute-force attacks.
\end{itemize}

\begin{figure}[h]
    \centering
    \includegraphics[width=0.9\linewidth]{images/locust1.png}
    \includegraphics[width=0.9\linewidth]{images/locust2.png}
\end{figure}